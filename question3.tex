\subsection*{(4.h) Changing the quantile to build 99\% CIs}

Replacing $0.975$ by $0.995$ in the $t$ quantile produces a \emph{99\%} confidence
interval for $\beta_2$:
\[
\left[\;\widehat\beta_2 - t_{0.995,\nu}\,\mathrm{SE}(\widehat\beta_2),\;
       \widehat\beta_2 + t_{0.995,\nu}\,\mathrm{SE}(\widehat\beta_2)\;\right],
\qquad \nu = n-2 = 48 .
\]
Under the classical linear model with normal, homoskedastic errors and fixed $x$,
this $t$–interval has \emph{exact} finite–sample coverage $p=0.99$ (conditional on $x$).
With $M=10{,}000$ replications, the reported statistic
\[
\texttt{Ratio}=\frac{1}{M}\sum_{m=1}^{M}\mathbf{1}\{\text{CI}_m\ \text{contains}\ \beta_2\}
\]
is the sample mean of i.i.d.\ Bernoulli$(p)$ variables, so
\[
\mathbb{E}[\texttt{Ratio}]=0.99,\qquad
\mathrm{SE}(\texttt{Ratio})=\sqrt{\frac{p(1-p)}{M}}
=\sqrt{\frac{0.99\cdot0.01}{10{,}000}}\approx 0.0010 .
\]
By the CLT, a 95\% normal approximation interval for the empirical coverage is
\[
0.99 \pm 1.96\times 0.0010 \approx [0.988,\;0.992].
\]
Hence we would expect \texttt{Ratio} to be very close to $99\%$ (within about $\pm0.2$ percentage points).

\paragraph{Effect on the plot.}
Because $t_{0.995,\nu}>t_{0.975,\nu}$, the intervals become \emph{wider} (both endpoints move farther from
$\widehat\beta_2$). Consequently, almost all intervals (about $99\%$) cross the vertical line at the true value
$\beta_2=5$; only about $1\%$ fail to cover. Visually: many fewer “non-covering” intervals and uniformly wider
segments.

\subsection*{(4.i) Purpose of the simulation}

This exercise illustrates:
\begin{itemize}
  \item the meaning of a confidence interval as a \emph{procedure with coverage}: across repeated samples,
        a $1-\alpha$ CI contains the true parameter with probability $1-\alpha$ (here verified empirically by \texttt{Ratio});
  \item how the nominal level ($95\%$ vs.\ $99\%$) trades off \emph{coverage} and \emph{width} (higher coverage $\Rightarrow$ wider CIs);
  \item Monte Carlo sampling variability: with finite $M$, the empirical coverage fluctuates around the nominal level with
        variance $p(1-p)/M$, shrinking as $M$ grows (LLN/CLT);
  \item in the classical normal homoskedastic regression with fixed regressors, the $t$–interval achieves
        its \emph{exact} nominal coverage in finite samples, which the simulation confirms as $M$ becomes large.
\end{itemize}
